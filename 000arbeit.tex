% LaTeX-Vorlage zur Erstellung von Abschlussarbeiten an der FH Aachen
% Author: Sven Hinz

\documentclass[12pt,a4paper]{scrreprt}

% Paket für Umlaute:
\usepackage[utf8]{inputenc}       % Cross Platform
%\usepackage[ansinew]{inputenc}   % Windows
%\usepackage[latin1]{inputenc}    % Linux
%\usepackage[applemac]{inputenc}  % Mac

\usepackage[ngerman]{babel}       % Sprache: deutsch
\usepackage{amsmath}
\usepackage{amsfonts}
\usepackage{amssymb}
\usepackage{makeidx}
\usepackage{graphicx}
\usepackage{epstopdf}
\usepackage{kpfonts}
\usepackage[left=2cm,right=2cm,top=2.5cm,bottom=2.5cm]{geometry}

\author{Sven Hinz} % --> Eigenen Namen einfügen

\usepackage[plainheadsepline,headsepline]{scrpage2}
\usepackage{color}
\usepackage{setspace}
\usepackage[numbers,square]{natbib}
\usepackage{longtable}
\usepackage{listings}
\usepackage{rotating}
\usepackage{pdfpages}
\usepackage{caption}
\usepackage{subcaption}
\parindent 0pt
\usepackage{booktabs}
\usepackage[export]{adjustbox}


% Schriftart
\usepackage{helvet}
\renewcommand{\familydefault}{\sfdefault}
\setkomafont{chapter}{\sffamily \large}
\setkomafont{section}{\sffamily \normalsize}
\setkomafont{subsection}{\sffamily \normalsize}
\setkomafont{subsubsection}{\sffamily \normalsize}
\addtokomafont{caption}{\sffamily \small}
\usepackage{courier}

% Abstand zwischen Kopfzeile und Kapitelüberschrift
\renewcommand*{\chapterheadstartvskip}{\vspace*{-0.75\baselineskip}}

% Einstellungen der Kopf- und Fußzeile
\pagestyle{scrheadings}
\ihead[\sffamily \bfseries \upshape \headmark]{\sffamily \bfseries \upshape \headmark}
\chead[]{}
\ohead[]{}
\ifoot[]{}
\cfoot[]{}
\ofoot[\sffamily \pagemark]{\sffamily \pagemark}
\automark[]{chapter}
\renewcommand*{\chapterheadendvskip}{\vspace*{1\baselineskip}}

% Formeln
\usepackage{fleqn} % linksbündig
\setlength{\mathindent}{1.5cm} % Einrücktiefe

% Tabellen
\usepackage{multirow} % mehrzeiliger Text in einer Spalte
\renewcommand{\arraystretch}{2} % Zeilenabstand vergrößern
\setlength{\doublerulesep}{0.1mm} % Abstand der Doppellinien verkleinern
\usepackage{tabu}
\newcolumntype{C}{>{\centering\arraybackslash$}p{3cm}<{$}}

% Quellcode / Kommandozeileneingabe
\lstdefinestyle{BashInputStyle}{
  language=bash,
  basicstyle=\small\ttfamily,
  %numbers=left,
  %numberstyle=\tiny,
  %numbersep=3pt,
  frame=tb,
  columns=fullflexible,
  %backgroundcolor=\color{yellow!20},
  linewidth=0.9\linewidth,
  xleftmargin=0.1\linewidth
}

% Inhalt
\renewcaptionname{ngerman}{\contentsname}{Inhalt} % Umbenennung in Inhalt

% Quellenverzeichnis
\renewcaptionname{ngerman}{\bibname}{Quellenverzeichnis} % Umbenennung in Quellenverzeichnis

\usepackage[
  tocindentmanual,
  tocflat,
  tocbreaksstrict,
  toctextentriesleft,
]{tocstyle}

% Abkürzungsverzeichnis
%\usepackage[intoc]{nomencl}
%\let\abbrev\nomenclature
%\renewcommand{\nomname}{Abkürzungsverzeichnis}
%\setlength{\nomlabelwidth}{.25\hsize}
%\renewcommand{\nomlabel}[1]{#1 \dotfill}
%\setlength{\nomitemsep}{-\parsep}
%\makenomenclature

\usepackage[]{acronym}

\begin{document}
\setstretch{1.25}
\addtocontents{toc}{\linespread{1}}

% Einbinden der Textinhalte mit '\include{...}'
% Die Dateien mit den Textinhalten befinden sich im Ordner 'doc'

\begin{titlepage}
	%ab hier kleinere Raender, mehr bedruckbare Flaeche.

	\thispagestyle{empty}
	\newgeometry{a4paper, portrait, left=0cm, right=0cm, top=0.6cm, bottom=0cm, includefoot}

	% FH Logo
	\begin{flushright}
		\includegraphics[width=1.7cm]{./pic/FHAC.jpg}
	\end{flushright}

	\vspace{-2.5cm}

	% Kopfzeile mit Fachbereich ...
	\centering \bfseries \Large FH~Aachen \\
	\vspace{0.5cm}
	\normalsize Fachbereich\\
	Maschinenbau~und~Mechatronik \\
	Studiengang~Produktentwicklung

	\vspace{1cm}

	%\centering \bfseries Bachelorarbeit
	\centering \bfseries Masterarbeit

	\vspace{0.8cm}

	%Titel der Arbeit
	\centering \begin{minipage}[t]{17cm}
		\centering \bfseries \large Der~Titel~der~Arbeit\\ ist~zweizeilig
		\medskip
	\end{minipage}

	\vspace{1.5cm}

	%Name und Matrikelnummer
	%\vspace*{1cm}
	%\hspace*{6.8cm}
	\begin{minipage}[t]{9cm}
		\centering Vorname Nachname \\ Matr.-Nr.: 123456
	\end{minipage}

	\vspace{2.1cm}

	%Professor und Betreuer
	%\vspace*{4.7cm}
	%\hspace*{6.8cm}
	\centering \begin{minipage}[t]{9cm}
		\centering \begin{tabular}{ll}
			Referent: & Prof. Dr-Ing. ...\\
			Korreferent: & Prof. Dr.-Ing. ...\\
			%Externer Betreuer: & Dipl.-Wirt.-Ing\\
		\end{tabular}
	\end{minipage}

	\vspace{7cm}

	% Firmenlogo
	%\begin{flushleft}
	%\centering \hspace{-8cm}
	%\begin{minipage}[t]{5cm}
			%\includegraphics[width=5cm]{./pic/firmenlogo.jpg}
	%\end{minipage}
	%\end{flushleft}


	%Erstellungsdatum
	%\vspace{-4cm}
	%\begin{flushright}
	\centering %\hspace{8cm}
	\begin{minipage}[b]{5cm}
			\centering
			\today\\ %Datum\\
			%\vspace{1cm}
			%In Zusammenarbeit mit\\
			%Firma, Ort\\
			%\vspace{1cm}
			%vertraulich bis xx.xx.xx
	\end{minipage}
	%\end{flushright}

	%\today
	\restoregeometry
\end{titlepage}


\clearpage
\chapter*{Erklärung}\label{erklaerung}
\markboth{Erklärung}{Erklärung}
Ich versichere hiermit, dass ich die vorliegende Arbeit selbstständig verfasst und keine anderen als die im Literaturverzeichnis angegebenen Quellen benutzt habe.

\bigskip

Stellen, die wörtlich oder sinngemäß aus veröffentlichten oder noch nicht veröffentlichten Quellen entnommen sind, sind als solche kenntlich gemacht.

\bigskip

Die Zeichnungen oder Abbildungen in dieser Arbeit sind von mir selbst erstellt worden oder mit einem entsprechenden Quellennachweis versehen.

\bigskip

Diese Arbeit ist in gleicher oder ähnlicher Form noch bei keiner anderen Prüfungsbehörde eingereicht worden.

\vspace{1cm}
Aachen, \today %Monat Jahr

\vspace{7cm}
%\section*{Geheimhaltung}\label{geheimhaltung}

%{\large\textbf{Geheimhaltung}}\\


%Die vorliegende Arbeit unterliegt bis ... der Geheimhaltung. Sie darf vorher weder vollständig noch auszugsweise ohne schriftliche Zustimmung des Autors, des betreuendes Referenten bzw. der Firma ... vervielfältigt, veröffentlicht oder Dritten zugänglich gemacht werden.

%\clearpage
\chapter*{Danksagung}\label{danksagung}
\markboth{Danksagung}{Danksagung}
Danke.

% Inhaltsverzeichnis
\clearpage
\makeatletter
\renewcommand*{\@dotsep}{1} % Punktabstand einstellen
\makeatother
\tableofcontents

% Das erste Kapitel soll auf einer ungeraden Seite beginnen.
\cleardoublepage
\setstretch{1.25}

% Nicht benötigte Kapitel können auskommentiert werden
% Für zusätzliche Kapitel müssen weitere Dateien im Ordner 'doc' angelegt werden

\clearpage
\chapter{\textbf{Einleitung}}\label{einleitung}
%\addtocontents{toc}{\vspace{0.8cm}}


%\par\medskip


\section{Motivation und Aufgabenstellung}
%\addtocontents{toc}{\vspace{0.8cm}}



%% Beispiel für das Einfügen einer Abbildung

%\begin{figure}[h]
%	\centering
%		\includegraphics[width=0.8\textwidth]{pic/dateiname.png}
%	\caption{Beispielbild}
%	\label{fig:beispielbild}
%\end{figure}
%\vspace{7cm} % Abstand unter dem Bild


\newpage

\section{Vorgehensweise}
\addtocontents{toc}{\vspace{0.8cm}} % -> Abstand im Inhaltsverzeichnis

% Untersuchungsverlauf(pro Kapitel ein kurzer Absatz mit Verweis auf die Kapitelnummer)
 % Einleitung
\clearpage
\chapter{\textbf{Grundlagen}}\label{grundlagen}
%\addtocontents{toc}{\vspace{0.8cm}}

\section{Unterkapitel}\label{unterkapitel}
\addtocontents{toc}{\vspace{0.8cm}}

Hier folgt ein Beispiel für eine Formel:

% Formel
\begin{equation}\label{waermestrom}
\dot Q = \frac{dQ}{dt} = \lambda \frac{T_1-T_2}{\Delta x} A
\end{equation}

Wie in Gleichung \ref{waermestrom} zu erkennen ist, wird der Wärmestrom $\dot Q$ von der Wärmeleitfähigkeit $\lambda$, der Fläche $A$ und der Temperaturdifferenz $\Delta T = T_1-T_2$ zwischen den betrachteten Orten $\Delta x$ linear beeinflusst.

%% Zwei Abbildungen, die zusammen gehören

%\begin{figure}
%        \centering
%        \begin{minipage}[c]{0.45\textwidth}
%                \includegraphics[height=6.5cm]{pic/dateiname1.png}
%        \end{minipage}
%        \begin{minipage}[c]{0.45\textwidth}
%                \includegraphics[height=6.5cm]{pic/dateiname2.png}
%        \end{minipage}
%        \caption{Zwei Abbildungen}\label{fig:zwei_abb}
%\end{figure}

\clearpage
\chapter{\textbf{Kapitel 3}}\label{kap3}
\addtocontents{toc}{\vspace{0.8cm}}

\clearpage
\chapter{\textbf{Kapitel 4}}\label{kap4}
\addtocontents{toc}{\vspace{0.8cm}}

\clearpage
\chapter{\textbf{Kapitel 5}}\label{kap5}
\addtocontents{toc}{\vspace{0.8cm}}

\clearpage
\chapter{\textbf{Kapitel 6}}\label{kap6}
\addtocontents{toc}{\vspace{0.8cm}}

\begin{table}[htb]
\caption{Messergebnisse}
\label{tab:messung}
\centering
\begin{tabu}{l|[2pt]C|C|C}
Stellung & \frac{T_U}{^\circ C}  & \frac{T_c}{^\circ C} & \frac{\Delta T}{^\circ C}  \\
\tabucline[2pt]{-}
senkrecht (0°) & 27,3 & 69,8 & 42,5\\
\tabucline[0.5pt]{-}
waagerecht (90°) & 26,6 & 70,6 & 44,0\\
\end{tabu}
\end{table}

\clearpage
\chapter{\textbf{Kapitel 7}}\label{kap7}
\addtocontents{toc}{\vspace{0.8cm}}

\clearpage
\chapter{\textbf{Kapitel 8}}\label{kap8}
\addtocontents{toc}{\vspace{0.8cm}}

\clearpage
\chapter{\textbf{Kapitel 9}}\label{kap9}
\addtocontents{toc}{\vspace{0.8cm}}

\clearpage
\chapter{\textbf{Kapitel 10}}\label{kap10}
\addtocontents{toc}{\vspace{0.8cm}}

\clearpage
\chapter{\textbf{Zusammenfassung und Ausblick}}\label{zusammenfassung}
\addtocontents{toc}{\vspace{0.8cm}}


% Nachspann
\nocite{waermeatlas} % Quelle wird nicht im Text erwähnt -> Quellenverzeichnis
\nocite{dubbel2005}
% Weitere quellen müssen in 'bib/quellen.bib' eingetragen werden
% !!! -> BibTex ausführen! Sonst tauchen die Quellen nicht im Verzeichnis auf.

% Quellenverzeichnis
\clearpage
\bibliographystyle{unsrtdin}
\bibliography{./bib/quellen}
\addcontentsline{toc}{chapter}{Quellenverzeichnis}
%\addtocontents{toc}{\vspace{0.8cm}}

% Abkürzungsverzeichnis
\clearpage
\markright{Abkürzungsverzeichnis}
\clearpage
\chapter*{Abkürzungsverzeichnis}\label{abkuerzungsverzeichnis}
\begin{acronym}[YTM]
\setlength{\itemsep}{-\parsep}

\acro{gravitation}[$g$]{\hspace{1cm}Gravitation in Nähe der Erdoberfläche}
\acro{Nu}[$Nu$]{\hspace{1cm}Nußelt-Zahl}
\acro{nu_luft}[$\nu_{Luft}$]{\hspace{1cm}Kinematische Viskosität von Luft}
\acro{Pr}[$Pr$]{\hspace{1cm}Prandtl-Zahl}
\acro{Q}[$\dot Q$]{\hspace{1cm}Wärmestrom}
\acro{Ra}[$Ra$]{\hspace{1cm}Rayleigh-Zahl}
\acro{rho_luft}[$\rho_{Luft}$]{\hspace{1cm}Dichte von Luft}
\acro{temperatur}[$T$]{\hspace{1cm}Temperatur}
\acro{umgebungstemperatur}[$T_{\infty}$]{\hspace{1cm}Umgebungstemperatur}

\end{acronym}

%\addtocontents{toc}{\vspace{0.8cm}}

% Abbildungsverzeichnis
\clearpage
\addcontentsline{toc}{chapter}{Abbildungsverzeichnis}
\listoffigures
%\addtocontents{toc}{\vspace{0.8cm}}

% Tabellenverzeichnis
\clearpage
\addcontentsline{toc}{chapter}{Tabellenverzeichnis}
\listoftables
\addtocontents{toc}{\vspace{0.8cm}}

% Anhaenge
\addcontentsline{toc}{chapter}{Anhang}
\appendix
%\input{./app/Dateiname}
\chapter{Datenblätter}
\begin{enumerate}
      \item Datenblatt 1
      \item Datenblatt 2
\end{enumerate}

% Anhänge im Ordner 'app' ablegen

%\includepdf[pages=1-4]{./app/Datenblatt1.pdf} % Datei mit 4 Seiten
%\includepdf[pages=1]{./app/Datenblatt2.pdf} % Datei mit einer Seite

\chapter{Konstruktionszeichnungen}
\begin{enumerate}
      \item Seitenansicht
      \item Draufsicht
\end{enumerate}
%\includepdf[pages=1]{./app/Seitenansicht.pdf}
%\includepdf[pages=1]{./app/Draufsicht.pdf}

\end{document}
